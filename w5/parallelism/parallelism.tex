\documentclass{article}

\usepackage[utf8]{inputenc}
\usepackage{color}
\usepackage[left=1in,right=1in,top=1in,bottom=1in]{geometry}
\usepackage{amsmath, amssymb, amsthm}
\usepackage{textcomp}

\usepackage{enumerate}

\usepackage{fancyhdr}
\pagestyle{fancy}
\fancyhf{}
\fancyhead[R]{Caner Derici}
\fancyfoot[C]{\thepage}

% \usepackage{lipsum}

\newcommand{\HRule}{\rule{\linewidth}{0.5mm}}
\newcommand{\Hrule}{\rule{\linewidth}{0.3mm}}

\makeatletter% since there's an at-sign (@) in the command name
\renewcommand{\@maketitle}{
  \parindent=0pt% don't indent paragraphs in the title block
  
  {\Large \bf \@title}
  
  \Hrule%
    
  \textit{\@author \hfill \@date}
  \par
}
\makeatother% resets the meaning of the at-sign (@)

\title{Y790-32707 - Assignment 5: Parallelism Exercises}
\author{}
\date{Caner Derici}

\begin{document}
%\pagenumbering{gobble}

\maketitle% prints the title block

\section{Parallelism Exercise}

\paragraph{} Edit the following items to correct any errors in \textbf{parallelism}.

\begin{enumerate}
  \item Between 1891 and 1894, Jack London spent his time as a sailor,
    a waterfront loafer, and sometimes was a hobo.
    
    \fbox{Between 1891 and 1894, Jack London spent his time as a sailor,
    a waterfront loafer, and sometimes a hobo.}
  \item In addition to writing poems, Carl Sandburg sang folk songs
    and was collecting old ballads.

    \fbox{In addition to writing poems, Carl Sandburg was singing folk
      songs and collecting old ballads.}
  \item Robert Frost not only wrote poems but also an instructor of
    poetry.

    \fbox{Robert Frost not only wrote poems but also instructed
      poetry.}
  \item Before the age of thirty, Samuel Clemens
    \begin{itemize}
    \item worked as a printer
    \item piloted boats on the Mississippi
    \item newspaper reporter in Virginia City and San Francisco
    \end{itemize}

    \fbox{
      \parbox{\linewidth}{Before the age of thirty, Samuel Clemens
      \begin{itemize}
      \item worked as a printer
      \item piloted boats on the Mississippi
      \item reported to newspapers in Virginia City and San Francisco
      \end{itemize}
    }}
      
   \item Before Little Women was published, its author, Lousia May
     Alcott, nursed Union soldiers and edited a magazine for children.

     \fbox{Correct}
   \item Katherine Anne Porter enjoyed reporting for a newspaper and
     taught at several colleges in addition to writing novels and
     short stories.

     \fbox{\parbox{\linewidth}{Katherine Anne Porter enjoyed reporting
         for a newspaper and teaching at several colleges, in addition
         to writing novels and short stories.}}
   \item As a girl, Eudora Welty enjoyed golfing, baseball and
     bicycling; as an adult, after working for a radio station and a
     newspaper, she started to write fiction and taking photographs.

     \fbox{\parbox{\linewidth}{As a girl, Eudora Welty enjoyed golfing, playing baseball
       and bicycling; as an adult, after working for a radio station
       and a newspaper, she enjoyed writing fiction and taking
       photographs.}}
   \item Flannery O'Connor, confined to the family farm by chronic
     illness for most of her adult life, disciplined herself to sit
     down at her desk every day and working there for two hours
     whether she felt inspired or not.
     
     \fbox{\parbox{\linewidth}{Flannery O'Connor, confined to the
         family farm by chronic illness for most of her adult life,
         disciplined herself to sit down at her desk every day and
         work there for two hours whether she felt inspired or not.}}
   \item Edith Wharton, whose upper-class background relieved her from
     ever having to worry about money, is admired for her novels about
     New York society as well as for writing Ethan Frome, a brief
     novel set in New England.

     \fbox{\parbox{\linewidth}{Edith Wharton, whose upper-class
         background relieved her from ever having to worry about
         money, is admired for her novels about New York society as
         well as her brief novel set in New England for Ethan Frome.}}
   \item Poet Edna St. Vincent Millat lived as a young woman in New
     York City, where she found freedom to voice her opinions, set her
     own rules, and concentrate on her writing career.
     
     \fbox{Correct}
\end{enumerate}

\section{Paragraph Transition Exercises (handout)}

\begin{enumerate}
  
\item \textit{In the following passage, one or more words at the beginning of the
second paragraph have been deleted. Use a transitional word or phrase
to clarify the shift between the two paragraphs.}

\paragraph{} As children growing up in a small town, my brother and I were the only
ones whose father was ``different.'' He couldn't sing the national
anthem or remember the words of the Pledge of Allegiance and found it
difficult to comprehend the intricacies of football and baseball.
\paragraph{} \textbf{Nevertheless}, he was a very special parent. On rainy days he
was always waiting for us at the school door, rubbers in hand; if we
were ill he was there to take us home. He owrked in town and was
available to take us to music and dancing lessons or on little
drives. When I was a small child he planted beside my window a
beautiful oak tree grew to be taller than our home.

\hfill -- Janet Heller, ``About Morris Heller''

\item \textit{In the following passage, we have deleted the first sentence of
  the second paragraph and the first two sentences of the third. For
  each of those paragraphs write one or two opening sentences to
  clarify the transition from one paragraph to the next.}

  \paragraph{} Outside, in our childhood summers--the war. The summers of 1939 to
  '45. I was six and finally twelve; and the war was three thousand
  miles to the right where London, Warsaw, Cologne crouched huge,
  immortal under nights of bombs or, farther, to the left where our
  men (among them three cousins of mine) crawled over dead friends
  from foxhole to foxhole towards Tokyo or, terribly, where there were
  children (our age, our size) starving, fleeing, trapped, stripped,
  abondoned.
  \paragraph{} \textbf{(1)}. A shot would ring in the midst of our play, freezing us in the
  knowledge that here at last were the first Storm Troopers till we
  thought and looked--Mrs. Hightower's Ford. And any plane passing
  overhead after dark seemed pregnant with black chutes ready to
  blossom. There were hints that war was nearer than it
  seemed--swastikated subs off Hatteras or the German sailor's tattered
  corpse washed up at Virginia Beach with a Norfolk movie ticket in his
  pocket.
  \paragraph{} \textbf{(1)}. \textbf{(2)}. Our deadly threats were polio, being hit by a car, drowning in
  pure chlorine if we swam after eating. No shot was fired for a
  hundred miles. (Fort Bragg -- a hundred miles.) We had excess food
  to shame us at every meal, excess clothes to fling about us in the
  heat of play.

  \hfill -- Reynolds Price, \textit{Permanent Errors}
\end{enumerate}

%% \begin{thebibliography}{1}

%% \bibitem{www}
%% E. Başar, C. Derici, Ç. Şenol, \emph{WorldWithWeb: A compiler from world applications to JavaScript}, In proceedings of The Scheme and Functional Programming Workshop\hskip 1em plus 0.5em minus 0.4em\relax Boston, Massachusetts, 2009.

%% \end{thebibliography}


\end{document}
