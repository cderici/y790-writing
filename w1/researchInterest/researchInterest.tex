\documentclass{article}

\usepackage[utf8]{inputenc}
\usepackage{color}
\usepackage[left=1in,right=1in,top=1in,bottom=1in]{geometry}
\usepackage{amsmath, amssymb, amsthm}
\usepackage{textcomp}

\usepackage{enumerate}

\usepackage{fancyhdr}
\pagestyle{fancy}
\fancyhf{}
\fancyhead[R]{Caner Derici}
\fancyfoot[C]{\thepage}

% \usepackage{lipsum}

\newcommand{\HRule}{\rule{\linewidth}{0.5mm}}
\newcommand{\Hrule}{\rule{\linewidth}{0.3mm}}

\makeatletter% since there's an at-sign (@) in the command name
\renewcommand{\@maketitle}{
  \parindent=0pt% don't indent paragraphs in the title block
  
  {\Large \bf \@title}
  
  \Hrule%
    
  \textit{\@author \hfill \@date}
  \par
}
\makeatother% resets the meaning of the at-sign (@)

\title{Y790-32707 - Assignment 1: Brief Research Interest Statements}
\author{}
\date{Caner Derici}

\begin{document}
%\pagenumbering{gobble}

\maketitle% prints the title block

\section{General Audiance}

\paragraph{}

To understand certain behavioral changes under different effects, the
scientific principle of experimentation applies to the computer
software as well. The story of executing a program on a computer
starts with a human readable code in a certain programming
language. That code needs to be translated by another program, namely
a compiler, into another representation that a machine can read and
execute. We are working on a just-in-time (JIT) compiler, namely
Pycket, for the Racket programming language. Given a program written
in the Racket language, Pycket translates it to a representation that
will be evaluated in later phases again by Pycket to produce the final
result.

\paragraph{} Alternatively Pycket could use the machine representation
produced by Racket's own compiler instead of manually creating one to
proceed to the final result. Since the Racket's compiler is designed
exclusively for Racket, the representation produced by it is highly
optimized. The aim of our study is therefore to modify Pycket to use
the alternative representation to find out the effects of the certain
optimizations that are performed by Racket's compiler to the overall
performance of Pycket. As a consequence, this will allow us to
understand better the individual effects of distinct optimizations to
the performance of JIT compilers in the general case.


\section{Expert Audiance}

\paragraph{} Observing the runtime reaction of a tracing just-in-time (JIT)
compiler to the known compiler optimizations over different kinds of
programs could significantly improve both the fundamental
understanding and the runtime performance of the tracing JIT
compilation. Currently we are working on such a compiler, namely
Pycket, that is automatically generated using the RPython translator
to implement the Racket language. Given a Racket program, Pycket
expands the code and creates from the surface syntax an abstract
syntax tree (AST) to be evaluated again by the JIT compiler to produce
the final result. Our aim is to modify Pycket to generate an AST using
the Racket bytecode compiler, which provides highly optimized Racket
bytecode.

\paragraph{} Running Pycket with AST's generated from the surface syntax and from
the bytecode will allow us to compare and observe the runtime
performance improvements that the optimizations provide. Furthermore,
selectively enabling distinct optimizations in the Racket bytecode
compiler and observing the difference in the performance will
empirically show how each optimization affects the runtime on which
kinds of programs. This we believe significantly improve the
understanding of how the known compiler optimizations work with the
tracing JIT compilers in the general case.



%% \begin{thebibliography}{1}

%% \bibitem{www}
%% E. Başar, C. Derici, Ç. Şenol, \emph{WorldWithWeb: A compiler from world applications to JavaScript}, In proceedings of The Scheme and Functional Programming Workshop\hskip 1em plus 0.5em minus 0.4em\relax Boston, Massachusetts, 2009.

%% \end{thebibliography}


\end{document}
