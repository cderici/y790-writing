\documentclass{article}
\linespread{1.5}
\usepackage[utf8]{inputenc}
\usepackage{color}
\usepackage[left=1in,right=1in,top=1in,bottom=1in]{geometry}
\usepackage{amsmath, amssymb, amsthm}
\usepackage{textcomp}

\usepackage{enumerate}

\usepackage{fancyhdr}
\pagestyle{fancy}
\fancyhf{}
\fancyhead[R]{Caner Derici}
\fancyfoot[C]{\thepage}

% \usepackage{lipsum}

\newcommand{\HRule}{\rule{\linewidth}{0.5mm}}
\newcommand{\Hrule}{\rule{\linewidth}{0.3mm}}

\makeatletter% since there's an at-sign (@) in the command name
\renewcommand{\@maketitle}{
  \parindent=0pt% don't indent paragraphs in the title block
  
  {\Large \bf \@title}
  
  \Hrule%
    
  \textit{\@author \hfill \@date}
  \par
}
\makeatother% resets the meaning of the at-sign (@)
\date{Caner Derici}
\title{Y790-32707 - Assignment 6: Data Commentary}
\author{}


\begin{document}
%\pagenumbering{gobble}

\maketitle% prints the title block


%% ---- Location elements / summary statements


%% Table 4 shows survey r

%% As can be seen,

%% This very high percentage of misbehavior is especially alarming,

%% Another notable result is that

%% The least frequently reported misbehaviors were

%% It is worthwhile to note that these different

%% This problem will likely continue until

%% ----- Highlighting statements

%% - you can spot trends or regularities in the data.

%% - you can separate more impo rtant findings from
%% less impo rtant ones.

%% - you can make claims of appropriate strength.

%% AVOID

%% > simply repeating all the details in words.
%% > attem pting to cover all the information.
%% > claiming more than is reasonable or defensible.


%% attitude : I think

%% hedge : it is likely that

%% booster : clearly there is a need to

%% With the exception of those

%% ORGANIZATON -> general to specific

%% - comparative statements, instaed of numbers

%% ---------- Conclusion

%% - explanations / implications of the data

%% -- reasoning process that led to the conclusions

%% ?? unexpected results or unsatisfactory data

%% ---> may be due to fluctuations
%% ---> can be attributed
%% ---> can probably be accounted for by a defect in 
%% ---> is probably a consequence of weaknesses in the experimen tal design.
%% ---> The problem .... would seem to stem from the ...

%% ?? possible further research or possible future predictions

\vspace{0.35cm}

Table 14 shows the country rankings for publications in Elsevier
Journals between 1996 and 2010. As can be seen, United States (US) not
only produces more than one fourth of the total scientific information
produced, but also seems to be leading the overall scientific
progress, judging by the first glance to the total citations of its
publications and the substantial difference with the second country in
each category. One other notable observation is that China and Japan
have rather low citations per publication rankings, even though they
have substantially higher number of publications. This, being also
relatively true for countries like India and Russian Federation, may
suggest that those countries have large number of studies with smaller
interest groups like culture related social fields such as
anthropology or linguistics. On the other hand, a high citations per
document rank for a country may indicate a high number of studies in
hard sciences such as physics or computer science. However, neither
the number of publications nor the citations per document could
indicate the overall productivity or influence of a country to the
overall scientific progress. Because a closer look reveals for
instance, US has more than \%45 of its total citations coming from the
publications produced in the US, while Austria and Singapore, ranked
as 23th and 32th respectively, has more than \%85 of their citations
coming from the publications that are produced outside. This however,
doesn't suggest that Austria or Singapore has more leading power in
overall scientific progress than US, because the productivity itself
is not a factor in this data. The data needs to be normalized by the
number of researchers in the countries, in order to make it easier to
deduce the level of productivity of an individual country by
considering only the number of publications or h-index. The h-index in
particular is designed as an attempt to measure both the productivity
and impact of the studies of an individual scientist, by taking into
account the number of publications and the number of citations per
publication. However, without a normalization, it's natural to expect
a higher h-index from a large group of researchers than a smaller
one. Additionally, it's a known fact that the h-index can be
manipulated by the self-citations, which in turn suggests that a
higher h-index is not a strong indication of productivity or impact of
a country that has a high self-citation percentage. Furthermore, a
clear comparison of countries in the sense of productivity and impact
requires a more clear data layout that for instance makes it explicit
the criteria for which the countries are sorted. For instance,
China(2nd) is placed above UK(3rd), which may only be attributed to
the number of publications. However, the sorting clearly doesn't
depend solely on the number of documents, as can be seen from UK and
Japan, which are ranked as 3rd and 4th respectively, even though Japan
has \%1.5 more citable documents than UK. Overall, the one solid fact
still remains, which is that a single country has a strong monopoly on
the scientific development, even though the deficiencies described
above are eliminated. As it may be related to political and social
reasons, I believe that it is also related to the lack of a
conventional sense of working together. Therefore the progress seems
likely to remain a monopoly until the scientific community around the
globe establishes a strong sense of collaboration.


%% \begin{thebibliography}{1}

%% \bibitem{www}
%% E. Başar, C. Derici, Ç. Şenol, \emph{WorldWithWeb: A compiler from world applications to JavaScript}, In proceedings of The Scheme and Functional Programming Workshop\hskip 1em plus 0.5em minus 0.4em\relax Boston, Massachusetts, 2009.

%% \end{thebibliography}


\end{document}
