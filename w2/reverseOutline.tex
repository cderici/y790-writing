\documentclass{article}

\usepackage[utf8]{inputenc}
\usepackage{color}
\usepackage[left=1in,right=1in,top=1in,bottom=1in]{geometry}
\usepackage{amsmath, amssymb, amsthm}
\usepackage{textcomp}

\usepackage{enumerate}

\usepackage{fancyhdr}
\pagestyle{fancy}
\fancyhf{}
\fancyhead[R]{Caner Derici}
\fancyfoot[C]{\thepage}

% \usepackage{lipsum}

\newcommand{\HRule}{\rule{\linewidth}{0.5mm}}
\newcommand{\Hrule}{\rule{\linewidth}{0.3mm}}

\makeatletter% since there's an at-sign (@) in the command name
\renewcommand{\@maketitle}{
  \parindent=0pt% don't indent paragraphs in the title block
  
  {\Large \bf \@title}
  
  \Hrule%
    
  \textit{\@author \hfill \@date}
  \par
}
\makeatother% resets the meaning of the at-sign (@)

\title{Y790-32707 - HW \#2: Reverse Outlining}
\author{}
\date{Caner Derici}

\begin{document}
%\pagenumbering{gobble}

\maketitle

%% Create a reverse outline for one of the model you selected in last
%% week’s homework. Document the topic of each paragraph, and supporting
%% points in each paragraph if present. Then write an commentary on the
%% article’s structure, and discuss the overall organizing principle of
%% the paper. Upload all elements--your reverse outline, the article you
%% outlined, and your commentary--to Canvas as PDF file attachments
%% and/or as a text entry...

%% Be sure to include in both your reverse outline and your commentary
%% citation information for the article you chose to outline.

\begin{center}
  Reverse Outline of : \textbf{The Structure and Interpretation of the Computer Science Curriculum}
\end{center}

\section*{Abstract}

\begin{enumerate}
\item 1
\item 2
\end{enumerate}

\section{History and critique}

\begin{enumerate}
\item 3
\item 4
\item 5
\item 6
\item 7
\end{enumerate}

\section{Structure}

\subsection{Solving constraints}

\begin{enumerate}
\item 8
\item 9
\item 10
\item 11
\item 12
\item 13
\item 14
\item 15
\item 16
\item 17
\end{enumerate}

\subsection{Principles of programming}

\begin{enumerate}
\item 18
\item 19
\item 20
\item 21
\item 22
\end{enumerate}
  
\subsection{Principles of teaching}

\begin{enumerate}
\item 23
\item 24
\item 25
\item 26
\end{enumerate}

\section{Interpretation: functional versus object-oriented programming}

\begin{enumerate}
\item 27
\end{enumerate}

\subsection{Functional and object-oriented programming}

\begin{enumerate}
\item 28
\item 29
\item 30
\end{enumerate}

\subsection{The role of Scheme}

\begin{enumerate}
\item 31
\item 32
\item 33
\item 34
\item 35
\end{enumerate}

\subsection{Programming environments}

\begin{enumerate}
\item 36
\item 37
\item 38
\end{enumerate}

\section{Interpretation: teaching design principles}

\subsection{Structure and Interpretation of Computer Programs}

\begin{enumerate}
\item 39
\item 40
\item 41
\item 42
\item 43
\item 44
\end{enumerate}

\subsection{How to Design Programs}

\begin{enumerate}
\item 45
\item 46
\item 47
\item 48
\item 49
\item 50
\item 51
\item 52
\item 53
\item 54
\item 55
\end{enumerate}

\section{Experience and outlook}

\begin{enumerate}
\item 56
\item 57
\item 58
\item 59
\end{enumerate}


% CITATION


%% \begin{thebibliography}{1}

%% \bibitem{www}
%% E. Başar, C. Derici, Ç. Şenol, \emph{WorldWithWeb: A compiler from world applications to JavaScript}, In proceedings of The Scheme and Functional Programming Workshop\hskip 1em plus 0.5em minus 0.4em\relax Boston, Massachusetts, 2009.

%% \end{thebibliography}


\end{document}
