\documentclass{article}

\usepackage[utf8]{inputenc}
\usepackage{color}
\usepackage[left=1in,right=1in,top=1in,bottom=1in]{geometry}
\usepackage{amsmath, amssymb, amsthm}
\usepackage{textcomp}

\usepackage{enumerate}

\usepackage{fancyhdr}
\pagestyle{fancy}
\fancyhf{}
\fancyhead[R]{Caner Derici}
\fancyfoot[C]{\thepage}

% \usepackage{lipsum}

\newcommand{\HRule}{\rule{\linewidth}{0.5mm}}
\newcommand{\Hrule}{\rule{\linewidth}{0.3mm}}

\makeatletter% since there's an at-sign (@) in the command name
\renewcommand{\@maketitle}{
  \parindent=0pt% don't indent paragraphs in the title block
  
  {\Large \bf \@title}
  
  \Hrule%
    
  \textit{\@author \hfill \@date}
  \par
}
\makeatother% resets the meaning of the at-sign (@)

\title{Y790-32707 - HW \#2: Reverse Outlining}
\author{}
\date{Caner Derici}

\begin{document}
%\pagenumbering{gobble}

\maketitle

%% Create a reverse outline for one of the model you selected in last
%% week’s homework. Document the topic of each paragraph, and supporting
%% points in each paragraph if present. Then write an commentary on the
%% article’s structure, and discuss the overall organizing principle of
%% the paper. Upload all elements--your reverse outline, the article you
%% outlined, and your commentary--to Canvas as PDF file attachments
%% and/or as a text entry...

%% Be sure to include in both your reverse outline and your commentary
%% citation information for the article you chose to outline.

\begin{center}
  Reverse Outline of : \textbf{The Structure and Interpretation of the Computer Science Curriculum}
\end{center}

\section*{Abstract}

\begin{enumerate}
\item Despite the fame and glory of the book \textit{Structure and
  Interpretation of Computer Programs (SICP)} and its influence on the
  introductory computing courses, Scheme and functional programming
  have lost their good impression due to shortcomings of the book and
  the quaintness of Scheme.
\item Proposing an alternative approach for the role of functional
  programming in the first-year curriculum, and introducing a new
  book, and discussing how the approach provides a well preparation
  for a course on object-oriented programming.
\end{enumerate}

\section{History and critique}

\begin{enumerate}
\item The book SICP liberated the introductory courses from to be
  designed around the syntax of a famous programming language, to be
  focused on the study of important ideas in computing such as
  functional abstraction, data abstraction, streams, data-directed
  programming, etc.
\item Quickly both the course idea and the Scheme language became the
  first choice of many universities.
\item In early 90s, the influence of SICP and Scheme started to decay
  due to some complaints (e.g. instructors saying SICP is too hard, or
  programmers saying Scheme was too different than mainstream
  languages).
\item This paper addresses the complaints and tries to overcome the
  problems of SICP approach. Paper structure is outlined.
\end{enumerate}

\section{Structure}

\subsection{Solving constraints}

\begin{enumerate}
\item A computing curriculum should produce programmers and software
  engineers capable of adapting to the ever-evolving nature of the
  field.
\item It's difficult to impose this goal to the first-year course due
  to a range of unrelated constraints (e.g. Faculty colleagues want a
  specific industry language).
\item Some faculty demand that the first course should teach a
  language that'll be used in upstream courses.
\item First-year students (and/or parents) come with strong,
  preconvinced notions about programming and computing (e.g. latest
  tech industry trends disucced in a magazine).
\item The state of the first year students' education also poses some
  constraints.
\item Students come with a wide range of expectations, but almost
  everyone expects that the college will help them find internships
  and professional positions.
\item Satisfying the primary goal subject to these constraints is
  hard. We need to choose a programming language and introduce
  programming ASAP in the curriculum, but we also need to convey the
  choices with good reasons.
\item Proposing a solution by having a second look on the goal and
  constraints. Curriculum cannot be based on the latest industry
  trends, as it changes too fast. Still the students should be
  prepared for the industry. We need to concentrate on the
  first-summer (internship), and the last-year (interview for
  positions).
\item We need to concentrate on teaching the principles most of the
  time, and show how these principles apply to the real world during
  the second semester of the first year and last year.
\item Year should start with a heavy emphasis on principles and second
  semester should illustrate the use of these principles in a
  currently fashionable programming langauge.
\end{enumerate}

\subsection{Principles of programming}

\begin{enumerate}
\item We should identify the technical principles for the first-year
  programming course. Good program design habits should be taught.
\item Students must learn to read problem statements carefully
  (e.g. identify classes of data that play a role) and rewrite these
  into useful pieces (e.g. example data and program usage).
\item Students must learn to organize programs to match for example
  the identified classes of data. Small changes in the problem
  translate to small changes in the program (helps industry too, rapid
  change in requirements).
\item Students must learn to use and walk through the examples they've
  constructed before writing any code.
\item The last point in particular suggests that functional languages
  with their natural model-view separation are superior choices for
  this first year.
\end{enumerate}
  
\subsection{Principles of teaching}

\begin{enumerate}
\item It's a challange for first-year instructor to understand the
  teaching priorities. Typically the teaching of language constructs
  is explicit while the teaching of design principles remains
  implicit.
\item Conventional approach to teaching programming reverses the
  natural roles of data and control. We should let data drive the
  syllabus. This would force students to understand how to go from
  data to design explicitly, and pick up language constructs
  implicitly.
\item Most students are active learners. Examples should concentrate
  on the design instead of specific langauge constructs.
\item Summary. First course should focus on the principles of program
  design, and should avoid any kind of distractions from these
  principles (e.g. picking problems from a complex domain or use
  complex language).
\end{enumerate}

\section{Interpretation: functional versus object-oriented programming}

\begin{enumerate}
\item Returning to the choice of programming language. Suggesting that
  the first semester uses a simple functional langauge and the second
  semester uses an industrially fashionable language (e.g. Java).
\end{enumerate}

\subsection{Functional and object-oriented programming}

\begin{enumerate}
\item Functional and object-oriented programming share the desired
  curricular focus on data as the starting point for program design.
\item The computational model of a functional language is a minor
  extension of that of secondary school algebra. The model of
  object-oriented computation requires a bit mote sophistication, but
  it builds upon the model of functional langauges.
\item Using a functional language followed by object-oriented language
  is thus the natural choice.
\end{enumerate}

\subsection{The role of Scheme}

\begin{enumerate}
\item We picked Scheme langauge as the starting point.
\item Scheme's syntax is simple.
\item Scheme's semantics is easy to understand (the language is a
  generalization of high school algebra, students doesn't need to
  think about registers, stacks, memory cells, and other low-level
  concepts).
\item Scheme is safe. It detects and pinpoints the errors, giving a
  valuable feedback.
\item Scheme is dynamically typed. We don't need to explain type
  errors vs syntax errors. We can also informally introduce a type
  system as sets of values which is more intuitive to the students
  (and provides a better transition to the second semester object
  oriented langauge).
\end{enumerate}

\subsection{Programming environments}

\begin{enumerate}
\item It's not only a language choice, the programming environment
  also matters a lot.
\item Like the langauge, the first programming environment should be
  a lightweight and easy-to-use tool.
\item We believe that the lack of such a programming environment hurt
  the SICP approach of teaching and the functional community in
  general. For this we have produced an environment that supports
  teaching program design principles with Scheme.
\end{enumerate}

\section{Interpretation: teaching design principles}

\subsection{Structure and Interpretation of Computer Programs}

\begin{enumerate}
\item SICP covers many important program design ideas.
\item SICP suffers from not looking at the programmers' perspective
  (e.g. not explaining how and when programmers need a particular
  idea, why do they need it)
\item While covering important ideas in programming with examples,
  SICP leaves the program design and the organization of these ideas
  implicit.
\item SICP also suffers from selecting exercises that uses complex
  domain knowledge.
\item While these examples may be interesting to the experience
  students, they're not to the beginners. Students are spending a lot
  of time on the domain knowledge and often end up confusing domain
  knowledge and program design knowledge.
\item SICP is great to introduce important concepts in computer
  science, but it fails to recognize the role of the first course in
  the curriculum.
\end{enumerate}

\subsection{How to Design Programs}

\begin{enumerate}
\item Introducing the book that addresses SICP's deficiencies along
  four dimensions. 1) Explicit discussion on program design. 2) Series
  of teaching languages (subsets of Scheme) with different knowledge
  levels. 3) Exercises focus on the program design, only a few require
  any domain knowledge. 4) More accessible forms of domain knowledge
  than SICP.
\item Chapters and exercises in HtDP focuses on program design, only
  the ``extended exercises'' concern domain knowledge.
\item Explicit design knowledge is formulated as the design
  recipe. Introducing basic design habits that the design recipe
  enforces.
\item The book contains a series of design recipes for designing
  different kinds of abstractions over different kinds of data. Each
  design recipe especially discusses when to use a technique or mode.
\item The recipes also introduce a new distinction into program
  design: structural vs. generative recursion.
\item Compare the programming with structural and generative recursion
  over example functions 'insert' and 'kwik'.
\item Distinguishing the two forms of recursion and focusing on the
  structural case makes our approach scalable to the object-oriented
  (OO) world.
\item SICP fails to distinguish the two notion of recursion. Failure
  to recognize structural (recursive) reasoning causes it to omit the
  discussion of reasoning about classes of data, which is the essence
  of OO programming.
\item HTDP introduces the idea of iterative refinement for both
  programs and data separately, which in turn helps students produce
  complex programs systematically, combining the design recipes with
  the idea of refinement.
\item HTDP and SICP also vastly differ with regard to the treatment of
  language syntax. HTDP gradually increases the complexity of the
  language while also explicitly discussing the need to do so.
\item Finally, HTDP uses domain knowledge differently from SICP. It
  uses domain knowledge that is within reach of most students.
\end{enumerate}

\section{Experience and outlook}

\begin{enumerate}
\item HTDP approach has been implemented at colleges and high
  schools. College level (measurable) results are very strong.
\item High School teachers report similar success results. Female
  students like the HtDP curriculum exceptionally well.
\item HTDP has validated the usefulness of functional programming and
  functional programming languages in the first programming
  course. The two keys to success were to tame Scheme into simple
  teaching languages and to distill well-known functional principles
  of programming into generally applicable design recipes.
\item Hoping other functional communities can replicate this success
  in different contexts. Teaching programming in plain Erlang,
  Haskell, or ML implicitly will not do. We need to understand the
  role of functional programming in our curricula and the needs of our
  students.
\end{enumerate}


% CITATION


%% \begin{thebibliography}{1}

%% \bibitem{www}
%% E. Başar, C. Derici, Ç. Şenol, \emph{WorldWithWeb: A compiler from world applications to JavaScript}, In proceedings of The Scheme and Functional Programming Workshop\hskip 1em plus 0.5em minus 0.4em\relax Boston, Massachusetts, 2009.

%% \end{thebibliography}


\end{document}
