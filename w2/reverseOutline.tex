\documentclass{article}

\usepackage[utf8]{inputenc}
\usepackage{color}
\usepackage[left=1in,right=1in,top=1in,bottom=1in]{geometry}
\usepackage{amsmath, amssymb, amsthm}
\usepackage{textcomp}

\usepackage{enumerate}

\usepackage{fancyhdr}
\pagestyle{fancy}
\fancyhf{}
\fancyhead[R]{Caner Derici}
\fancyfoot[C]{\thepage}

% \usepackage{lipsum}

\newcommand{\HRule}{\rule{\linewidth}{0.5mm}}
\newcommand{\Hrule}{\rule{\linewidth}{0.3mm}}

\makeatletter% since there's an at-sign (@) in the command name
\renewcommand{\@maketitle}{
  \parindent=0pt% don't indent paragraphs in the title block
  
  {\Large \bf \@title}
  
  \Hrule%
    
  \textit{\@author \hfill \@date}
  \par
}
\makeatother% resets the meaning of the at-sign (@)

\title{Y790-32707 - HW \#2: Reverse Outlining}
\author{}
\date{Caner Derici}

\begin{document}
%\pagenumbering{gobble}

\maketitle

%% Create a reverse outline for one of the model you selected in last
%% week’s homework. Document the topic of each paragraph, and supporting
%% points in each paragraph if present. Then write an commentary on the
%% article’s structure, and discuss the overall organizing principle of
%% the paper. Upload all elements--your reverse outline, the article you
%% outlined, and your commentary--to Canvas as PDF file attachments
%% and/or as a text entry...

%% Be sure to include in both your reverse outline and your commentary
%% citation information for the article you chose to outline.

\section{Reverse Outline}

% CITATION

\section{Commentary}


%% \begin{thebibliography}{1}

%% \bibitem{www}
%% E. Başar, C. Derici, Ç. Şenol, \emph{WorldWithWeb: A compiler from world applications to JavaScript}, In proceedings of The Scheme and Functional Programming Workshop\hskip 1em plus 0.5em minus 0.4em\relax Boston, Massachusetts, 2009.

%% \end{thebibliography}


\end{document}
