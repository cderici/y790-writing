\documentclass{article}

\usepackage[utf8]{inputenc}
\usepackage{color}
\usepackage[left=1in,right=1in,top=1in,bottom=1in]{geometry}
\usepackage{amsmath, amssymb, amsthm}
\usepackage{textcomp}

\usepackage{enumerate}

\usepackage{fancyhdr}
\pagestyle{fancy}
\fancyhf{}
\fancyhead[R]{Caner Derici}
\fancyfoot[C]{\thepage}

% \usepackage{lipsum}

\newcommand{\HRule}{\rule{\linewidth}{0.5mm}}
\newcommand{\Hrule}{\rule{\linewidth}{0.3mm}}

\makeatletter% since there's an at-sign (@) in the command name
\renewcommand{\@maketitle}{
  \parindent=0pt% don't indent paragraphs in the title block
  
  {\Large \bf \@title}
  
  \Hrule%
    
  \textit{\@author \hfill \@date}
  \par
}
\makeatother% resets the meaning of the at-sign (@)

\title{Y790-32707 - HW \#2: Reverse Outlining - Commentary}
\author{}
\date{Caner Derici}

\begin{document}
%\pagenumbering{gobble}

\maketitle

%% Create a reverse outline for one of the model you selected in last
%% week’s homework. Document the topic of each paragraph, and supporting
%% points in each paragraph if present. Then write an commentary on the
%% article’s structure, and discuss the overall organizing principle of
%% the paper. Upload all elements--your reverse outline, the article you
%% outlined, and your commentary--to Canvas as PDF file attachments
%% and/or as a text entry...

%% Be sure to include in both your reverse outline and your commentary
%% citation information for the article you chose to outline.

\begin{center}
  Reverse Outline of : \textbf{The Structure and Interpretation of the Computer Science Curriculum \cite{sicc2004}}
\end{center}

%% article structure

%% overall organizing principle

%% Does every paragraph relate back to your main idea?

%% Where might a reader have trouble following the order of your ideas?

%% Do several of your paragraphs repeat one idea?

%% Does one paragraph juggle several topics?

%% Are your paragraphs too long? Too short?


%% Is there a recognizable topic sentence?

\paragraph{} This paper observes several deficiencies of a famous book SICP and its
effect on the design of the computer science curriculum, particularly
focusing on the first year programming course. Organization of the
paper is very clear. It presents the problem in the first section. The
second section lays out the constraints on the problem. Third section
explains the principles of the solution. Fourth section introduces the
new book designed around these principles, and compares it to
SICP. Fifth section concludes the text.

\paragraph{} The section titles relate to the main idea and the title of the paper,
and looking at the individual sections and subsections, each paragraph
relates back to the main idea of the corresponding section and/or
subsection. There's not any particular piece in text that a reader
might have some trouble following the main idea.

\paragraph{} All paragraphs seem to be concerned with only one point. There are
couple of paragraphs where it seems like there are two ideas, but
usually one of them is a supporting argument for the main point. There
are no two paragraphs making the same point. And they are more or less
the same in size. Most of them have a recognizable topic sentence as
the first sentence and supporting arguments afterwards.

% CITATION


\begin{thebibliography}

\bibitem{sicc2004} Felleisen, M., Findler, R. B., Flatt, M., &
  Krishnamurthi, S. (2004). The structure and interpretation of the
  computer science curriculum. Journal of Functional Programming,
  14(04), 365-378.


%% \bibitem{www}
%% E. Başar, C. Derici, Ç. Şenol, \emph{WorldWithWeb: A compiler from world applications to JavaScript}, In proceedings of The Scheme and Functional Programming Workshop\hskip 1em plus 0.5em minus 0.4em\relax Boston, Massachusetts, 2009.

\end{thebibliography}


\end{document}
