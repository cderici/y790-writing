\documentclass{article}

\usepackage[utf8]{inputenc}
\usepackage{color}
\usepackage[left=1in,right=1in,top=1in,bottom=1in]{geometry}
\usepackage{amsmath, amssymb, amsthm}
\usepackage{textcomp}
\usepackage{setspace} 
\usepackage{enumerate}

\usepackage{fancyhdr}
\pagestyle{fancy}
\fancyhf{}
\fancyhead[R]{Caner Derici}
\fancyfoot[C]{\thepage}

% \usepackage{lipsum}

\newcommand{\HRule}{\rule{\linewidth}{0.5mm}}
\newcommand{\Hrule}{\rule{\linewidth}{0.3mm}}

\makeatletter% since there's an at-sign (@) in the command name
\renewcommand{\@maketitle}{
  \parindent=0pt% don't indent paragraphs in the title block
  
  {\Large \bf \@title}
  
  \Hrule%
    
  \textit{\@author \hfill \@date}
  \par
}
\makeatother% resets the meaning of the at-sign (@)

\title{Y790-32707 - HW \#3: Controversy in Computer Science}
\author{}
\date{Caner Derici}

\begin{document}
%\pagenumbering{gobble}
\doublespacing

\maketitle

%% IDEAS

\section{intro}

From popular culture to the top tier science labs around the world,
one controvercy in computer science was always getting a huge
attention everywhere, namely Artificial Intelligence (AI). Rooted at
the very birth of modern computers, and attracted attention from the
very pioneers of computer science, AI kept its piquancy over the
decades among not only the scientists but also all kinds of
imaginative people around the globe.

Artificial intelligence (AI) is a subfield of computer science, that
basically tries to mimic the human intelligence on a computer. In
other words, it tries to make the computers think, reason and create
new information using the information at hand. The field began its
journey in the early 50s as a philosophical debate. However, the
proliferation of modern computer technologies in the 70s made it
possible to realize most of the controversies into the actual
implementations for researchers to experiment upon.

While the idea of machines being intelligent means for example a
computer is going to be able to play chess (i.e. devise a strategy in
a game), it also means that a computer may as well make a crucial
decision on its own, which can be both magnificent and terrifying at
the same time. And judging by the role of computers in both past
decades' and today's societies, it's easy that the idea of thinking
computers provokes all kinds of controversies, such as computers
taking over the world, making decisions that effect the lives of
billions, or simply acting as a real person on the social media.

The big question was and still is ``Can machines think?'', posed for
the first time by Alan Turing in his famous article titled ``Computing
machinery and intelligence'' \cite{turing}. Turing approached the
problem by proposing an experiment to determine whether the given
entity is intelligent, that is now famously known as the Turing Test. 


the controversy


. Philosophy of mind. Consciousness.

what is AI?

\section{can machines think?}

Alan TUring, brief history of AI

We like to believe that Man is in some subtle way
superior to the rest of creation. It is best if he can be shown to be necessarily superior, forthen there is no danger of him losing his commanding position.


Edsger Dijkstra summarized this latter view brilliantly:
…Alan M. Turing thought about criteria to settle the question of
whether Machines Can Think, a question of which we now know that it is
about as relevant as the question of whether Submarines Can Swim.

\section{can a computer be conscious?}

What if computers became so intelligent that they
are able to exhibit emotions and feelings

\section{can/will computers take over the world?}

devastation unmanned aerial
vehicles can bring but what if instead of a person “behind the wheel”
of a drone, it was strictly a computer making these decisions?

\section{conclusion}

- right now, no, but the human brain is a deterministic machine, so
eventually the mechanisms of how we actually think will be discovered
and simulating that behavior with a machine will be possible.

- consciousness is not a well defined concept, so we're not even sure
what the question means, let alone answer it.

- computers themselves won't do that, but the people who program their
egos into those machines could. But maybe the machines will take over
some of the jobs of humans (such as marketing specialist economists
trying to predict the market behavior).



Artificial Intelligence (AI) is not posing too much of a problem right
now, but it definitely concerns scientists with its potential.  After
all, we are trying to get computers to become intelligent, near
thinking entities.   – would we no longer be able
to justify their constant servitude to the human race?  But even that
is getting way ahead of problems that we are going to face much
sooner.  We have all heard about the     Oh, and of course, we need to keep their egos in
check lest they decide to take over the world.

A big one comes from philosophy of mind: can a computer be conscious?
Views are split: some people heavily believe they can, some people
believe they can't and some (probably on the CS rather than philosophy
side of things) believe the question isn't even meaningful. The
inimitable 

AI is still a popularly fascinating topic that captures the
imagination and, of course, popular controversy. Aside from
philosophical considerations, some people (including AI luminaries
like Stuart Russell) view the possibility of general artificial
intelligence as incredibly dangerous for the human race. And even
existing systems, far from "general" intelligence, have controversies
of their own—what do we do, for example, with an algorithm that spits
out biased results given biased data? Could this be a form of
unlawful/unethical discrimination?

PRIVACY

Concerns about privacy have been everywhere lately, and that is
because the issue is very complex.  First of all, people are concerned
about the lengths the U.S. government has gone to keep Americans safe.
This has included large amounts of data collection on cell phone and
Internet communications – like the NSA collecting more than 200
million text messages a day and putting tracking software on more than
100,000 computers around the world.  Before this constant stream of
new, crazy information began to be leaked by Eric Snowden, many
Americans were in the dark about the truly amazing technology the
government was using.  The President has made it clear after the huge
public outcry that the NSA will reform its methods of data collection,
but maybe the government will just get better at hiding it?  We may
never know.

This is all without even discussing the privacy issues sitting in the
palms of our hands.  Our computers and phones, along with the websites
we go on every day (see: Google, Facebook), are constantly collecting
information on what we see and do.  Unless your last name is Obama or
Gates, no one really cares what you, individually, are doing online,
but all of that information collected in mass is very valuable to
advertisers and large companies – all they want is to target their ads
to you in creative, but sometimes creepy ways.

One of the biggest trends in computer science both in industry and in
academia is the somewhat nebulous notion of "big data": tools and
methods that gather and analyze immense amounts of data, often about
individual people. This leads to both broad conclusions about the
whole population and narrower conclusions about single individuals
like you.

Big data technology has put an unprecedented amount of power in the
hands of corporations and, especially, the government. Where in the
past the government was limited in how they could conduct
surveillance—following a single person was expensive, so they could
only do it so much—now it can effectively follow everyone, everywhere,
all the time. Okay, that might be a bit of an exaggeration, but it's
close: as recent revelations from Snowden, Wikileaks and others have
confirmed, the government was actively monitoring vast swathes of
internet communication. A decade or two ago this would have been
impractical, but with new technology that can find needles in
unthinkably large haystacks this sort of tracking is incredibly
powerful.

PROGRAMMING?

20 controversial programming opinions

http://programmers.blogoverflow.com/2012/08/20-controversial-programming-opinions/

%% article structure
%% overall organizing principle
%% Does every paragraph relate back to your main idea?
%% Where might a reader have trouble following the order of your ideas?
%% Do several of your paragraphs repeat one idea?
%% Does one paragraph juggle several topics?
%% Are your paragraphs too long? Too short?
%% Is there a recognizable topic sentence?



\begin{thebibliography}

%% \bibitem{sicc2004} Felleisen, M., Findler, R. B., Flatt, M., &
%%   Krishnamurthi, S. (2004). The structure and interpretation of the
%%   computer science curriculum. Journal of Functional Programming,
%%   14(04), 365-378.

\bibitem{turing} Turing, A. M. (1950). Computing machinery and intelligence. Mind, 59(236), 433-460.
%% %% \bibitem{www}
%% %% E. Başar, C. Derici, Ç. Şenol, \emph{WorldWithWeb: A compiler from world applications to JavaScript}, In proceedings of The Scheme and Functional Programming Workshop\hskip 1em plus 0.5em minus 0.4em\relax Boston, Massachusetts, 2009.

\end{thebibliography}


\end{document}
