\documentclass{article}

\usepackage[utf8]{inputenc}
\usepackage{color}
\usepackage[left=1in,right=1in,top=1in,bottom=1in]{geometry}
\usepackage{amsmath, amssymb, amsthm}
\usepackage{textcomp}

\usepackage{enumerate}

\usepackage{fancyhdr}
\pagestyle{fancy}
\fancyhf{}
\fancyhead[R]{Caner Derici}
\fancyfoot[C]{\thepage}

% \usepackage{lipsum}

\newcommand{\HRule}{\rule{\linewidth}{0.5mm}}
\newcommand{\Hrule}{\rule{\linewidth}{0.3mm}}

\makeatletter% since there's an at-sign (@) in the command name
\renewcommand{\@maketitle}{
  \parindent=0pt% don't indent paragraphs in the title block
  
  {\Large \bf \@title}
  
  \Hrule%
    
  \textit{\@author \hfill \@date}
  \par
}
\makeatother% resets the meaning of the at-sign (@)

\title{Y790-32707 - Assignment 4: Flow Techniques}
\author{}
\date{Caner Derici}

\begin{document}
%\pagenumbering{gobble}

\maketitle% prints the title block


\section{General Audiance - NEW}

\paragraph{} The story of executing a computer software starts with a human
readable code in a certain programming language. This software code
needs to be translated by another software, namely a compiler, into a
type of code that a machine can read and execute. We are working on
such a compiler, in particular a just-in-time (JIT) compiler named
Pycket, that translates software written in the Racket programming
language. Given a software written in Racket, Pycket translates it to
a type of code that can be read and executed in later phases to
produce the final result.

\paragraph{} Instead of translating the human readable software in Racket language,
Pycket can be modified to translate machine readable code produced by
Racket's own compiler, which is exclusively designed for Racket. Thus
Pycket can translate a highly optimized version of the Racket
software, which was originally untouched in its human readable
form. The aim of our study is therefore to modify Pycket to translate
the optimized Racket code and compare the performances of the two
versions of Pycket. This will reveal the effects of the Racket
optimizations to the overall performance of Pycket. As a consequence,
this will also allow us to understand better the individual effects of
the distinct compiler optimizations to the performance of JIT
compilers in general.


\section{Book Exercise}

\paragraph{} Palace revolts and popular revolutions plagued seven out of eight
reigns of the Romanov line after Peter the Great. In 1722, Peter
terminated the principle of heredity, and made achievement by merit
the basis of succession. This resulted in many tsars not appointing a
successor before dying, including Peter. Succession not dependent upon
authority resulted in the boyars' regularly disputing who was to
become sovereign. Czarina Anna appointed Ivan VI, who was less than
two months old, but Elizabeth, the daughter of Peter the Great,
defeated Anna and ascended to the throne in 1741. Paul I codified the
law of succession in 1797 to be male primogeniture. However,
conspirators strangled him, one of whom was probably his son,
Alexander I.

%% \begin{thebibliography}{1}

%% \bibitem{www}
%% E. Başar, C. Derici, Ç. Şenol, \emph{WorldWithWeb: A compiler from world applications to JavaScript}, In proceedings of The Scheme and Functional Programming Workshop\hskip 1em plus 0.5em minus 0.4em\relax Boston, Massachusetts, 2009.

%% \end{thebibliography}


\end{document}
